The RAM Legacy database collates stock assessment data and estimates derived from a variety of methods. Assessments may be based on integrated statistical models using length and age data which %require fixing or the use of priors for difficult to estimate parameters such as the steepness of the stock-recruitment relationship, and 
estimate reference points as part of the fitting process, age-based models based on  virtual population analysis where reference points are estimated in post-processing, or biomass dynamic models where assumptions related to density dependence (i.e. growth, mortality and recruitment) are modelled by a production function. Despite these difference, all models assume a production function, either explicitly as in the case of biomass dynamic or implicitly in the case of age-based models. The production function provides the basis of maximum sustainable yield ($MSY$) based on reference points \citep{sissenwine1987alternative}. Age-based models provide estimates of spawning stock biomass (SSB) and instantaneous fishing mortality, while biomass dynamic model estimates correspond to exploitable biomass and harvest rate.

Therefore time-series of biomass and exploitation in the RAM Legacy database are provided in a variety of ways. Trends in biomass are based either on $SSB$ or total biomass $B$, while trends in exploitation rate are based on either instantaneous fishing mortality ($F$) or annual exploitation rate ($U$). %Relative trends are provided with respect to $MSY$ or management reference points. The estimates of reference points may be provided as part of the original assessment or estimated after compilation, by the database maintainers, by fitting a surplus production model. 

\iffalse
This can introduce bias when using them to evaluate the performance of assessment methods. Therefore we undertook a hierarchical approach based on the quantities available by stock; namely, the number in brackets gives the number of cases 

\begin{itemize}
 \item total biomass, exploitation rate and $MSY$ reference points (5)
 \item total biomass, instantaneous fishing mortality and $MSY$ reference points (37)
 \item spawning stock biomass, instantaneous fishing mortality and $MSY$ reference points (67)
\end{itemize}
\fi

Trends in  $F/F_{MSY}$, $B/B_{MSY}$, and $Catch/MSY$ are shown in figures \ref{fig:ts-f}, \ref{fig:ts-s} and \ref{fig:ts-c} respectively. Individual stocks are represented by the faint lines, the median trend by the thick line,interquartiles by the thick dashed lines and the $90^{th}$ percentiles by the thin dashed lines.  There has been a gradual increase in catches peaking around 1990, after which catches declined. Fishing mortality also peaked in 1990 and stayed around the $F_{MSY}$ until today, while the stocks have continued to decline. A noticeable feature is that some stock has shown wide variability while others have shown smooth trends. The median shows that stock have declined since the start of the series in the 1950s until 2000, after which the stocks stabilised. There is however a high degree of variability on a stock-by-stock basis. This is because fishing mortality had increased until the 1990s after which it varied just below the $F/F[MSY]$ level. This due to the adoption of target reference points based on $MSY$ by many management bodies after the adoption of the Precautionary Approach. Currently catches are less than $MSY$  and yields follow the general trends in biomass and fishing mortality, however yield and biomass are lagged biomass responding to catch a few years later. Fishing mortality is variable, reflecting that management is generally based on catch and that biomass is also influenced by process error, e.g. variability in year-class strength.  
