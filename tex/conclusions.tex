\begin{itemize}
\item Previous studies have used packages that implement the same basic algorithm with extra features such as ways for setting priors, the shape of the production function and heuristics for depletion. They have shown that the choices made are essential to get a good estimate of depletion in the final year \citep{rosenberg2014developing}. However, even for data rich stocks productivity depends on difficult to estimate parameters such as the steepness of the stock recruitment relationship and natural mortality.  
\item We used JABBA a biomass based stock assessment method that is used to provide advice for a number of data rich stocks. This allowed us to evaluate the value-of-infomation across the data poor to rich spectrum.
\item $F/F_{MSY}$ poorly estimated due to mining and lags.
\item ROC curves, distance of threshold from [0,1] shows how good at classification of state, AUC shows how good at ranking.
\item Data have no affect
\item Snapshot, but then need to know depletion, i.e. priors are developed by region
\item Can not validate, or use in MSE as a feedback controller
\item Solution is to collect indices of abundance
\item Develop MPs \citep{fischer2020hcr}
%In this paper we evaluated the ability of two COMs, sraplus and CMSY+, the estimate key fishery references points, B/Bmsy and F/Fmsy, using varyingly informative levels of data. Our results corroborate those of Rosenburg et. al (2016), and indicate that one should not run these models using the default settings setting the depletion priors and the shape of the production function is essential to get a good estimate of depletion  in the final year. Even with tuning of the default heuristics, the estimates of biomass reference points were overestimated for SRA+ and  under estimated for CMSY+ (Figure 5, [this is the figure of the box plot across methods that we should have for the last year performance on all 48 stocks wrt to BMSY and FMSY]). Bias and precision improves as knowledge increases. If similar priors are used for both implementations, the results are equivalent. This is  not surprising as the underlying assumptions of both are based on the stock reduction algorithm (Kimura et. al. 1993). Differences are due to the default catch rules, and the use of priors for life history and/ depletion in how these models are used.

%Both implementations unsurprisingly perform well if one had a good knowledge of  the fishery for a stock. Simple indicators like average length of individuals in the catch or the history of a fishery (e.g. for an artisanal where the introduction of mechanization is known) could inform us about current (Reference). The catch rules used had large confidence intervals , yet could estimate relative biomass levels fairly well (Table 4). In addition, estimates of MSY were also quite well estimated  (see Martell and Forese 2012). These methods could potentially provide guidance on target yield level, however, more needs to be known about the history of the fishery if  current depletion levels are to be estimated.

%Good initial depletion estimates were also needed. If the time series are short this is true for most assessment models (e.g.  Synthesis (Methot and Wetzel (2013) or Jabba (Winker et. al. 2018)), although initial depletion is difficult to estimate. If a good estimate of initial depletion is available the methods do substantially better, as shown by the  the biomass models/expert biomass models versus naïve models/expert naïve models, comparisons 

%The more information there is on the life history of a stock, and the more data for calibration, e.g.  effort or indices of abundances even for a short the models  perform better, as is illustrated with the improved precision  more information is provided (Figure 5). In other words a biomass dynamic model with an index of abundance these models will perform much better than a COM.

%Approach currently underdevelopment (Amoroso et. al. 2019 and Ovando et. al. 2019) indicate that using external sources of information like Swept Area Ratio (SAR) or a Fishery Management Index (FMI, Melnychuck et. al. 2019) could provide information on current depletion. This would be invaluable in cases where where is limited information on the history of the fishery. 

%With regard to assigning individual stocks  to the different states of biomass, the methods achieved close to 80% accuracy  (Figure 7 and 8), however, categorization with respect to fishing mortality was poor (Figure 3). This should is not surprising as there is lesser information on mortality than biomass from catch (Rosenburg et. al. 2016). For example catch could be relatively constant in a fishery where entry was unrestricted and the stock declining. 

%Whilst these methods may be good for categorising data poor stocks, there is no  substitute for field programs and direct data relevant to assessments (survey data or length data). The better the information, the better we can truly understand the dynamics of these fisheries and populations being examined, and reduce the risk to society. In many parts of the world, however, this may not be possible as such programs are expensive to run, and in Asia, S. and Central America, and Africa other priorities take precedence (Ref). However, the long term cost of depleting a resource may be greater than the investment required to obtain better data. Also the objective of fish stock assessment is to provide objective feedback on the response of a stock to management measures, if the main inputs to an assessment method depend on knowledge of that response then advice may not be robust. 

%As shown in the analysis, these approaches have some potential in regional classification even if the individual assignments are incorrect. As a framework for advice (Figure 4 and 5) indicate what maybe happening at a regional scale quite accurately, and can provide broad advice for a region and globally. The algorithms used for classification need to be clearly documented, so one can replicate the methodology globally using similar rules, something that has been missing for a while on the global scale studies run currently (Froese et. al. 2018). In addition, the potential of developing priors from external sources make this approach attractive for many reasons as it can constantly be improved as more data is collected on the ground. 

%Finally, SDG 14.4.1, a UN sustainable development goal states: “ By 2020, effectively regulate harvesting and end over-fishing, illegal, unreported and unregulated fishing and destructive fishing practices and implement science-based management plans, in order to restore fish stocks in the shortest time feasible, at least to levels that can produce maximum sustainable yield as determined by their biological characteristics”. This methodology as shown by SRA+ (using external covariates to determine depletion) or expert based rules (CMSY+ and SRA+) can be developed on a country or regional level so reporting on this metric is attainable, and universally standardized so we can compare across countries or regions in the world as to how they are doing with respect to their over-fishing targets. 
%A possible caveat is that the ICES stocks, which were the basis of this study, have been managed by an MSY approach since 2008, this means that the catch heuristics used are inappropriate, since management is actively manipulating the relationship between biomass and catch, and it is difficult to generalise the results on the ICES test set used here to stocks from other regions. Most scenarios evaluated use data are from the left-hand limb of the production function but in stocks for which COMs will be applied then datasets data will be from the right-hand limb, so it maybe difficult to generalize from these results. The dataset was quality controlled , however, and so serves as a good basis for comparing alternative platforms, the real focus of this paper and their generalized trends. 

%Another issue is validation since catch-only methods do not use auxiliary data they can not be validated using observations (Kell et al., 2021). This is a problem as the methods are assumption driven, and assume the catch data known without error. There  are therefore no methods to make an objective choice between alternative COM runs when conducting sensitivity tests, since diagnostics such as goodness of fit based on residuals, retrospective analysis and cross validation are not applicable. This is a problem with these methods in general, but again in absence of better data, these are the only tools available for inferring status on large number of stocks without any formalized assessment.

%The results of this paper support the concerns that were raised about catch only methods in some of the previous evaluations (Free et. al. 2020, Rosenburg et. al. 2016). In particular, it appears especially clear that the default settings for these models even when informed by life history metanalyses such as FishBase should not be expected to provide reliable estimates of individual’s species dynamics, of regional level stock evolution over time that FAO aims to describe, nor can either of the models be relied upon in their default settings to provide meaningful estimates of SDG14.4.1 indicator that UN needs to track sustainable development goal related to overfishing. SRA+ model seems to be positively biased, and is likely to say that a higher proportion of fish stocks are harvested sustainably than is really the case.  But while SRA+ model is more optimistic, the CMSY+ model’s estimates of SDG14.4.1 can be both more pessimistic depending on the additional information that the model is provided with in terms of priors.  

%Performance of both models improves when extra knowledge is being supplied in terms of informative priors. Unsurprisingly, the most valuable in terms of improving overall model performance are informative priors of initial depletion combined with an excellent index of abundance.  A natural recommendation follows that if alternative models are not applicable due to the lack of data, then investing in eliciting knowledge on the initial state of the stock and research that can be used to construct a plausible index of abundance would be worthwhile.  From this investigation, although it has various limitations, further broad conclusions can be drawn that are probably fairly robust. While, SRA+ is marginally superior to CMSY+ method if both sensitivity and specificity of the method are important, we recommend that both these methods are continued to be developed and tested and alternative model options are explored. Besides, both models perform similarly and are equivalent when the assumptions driving them are the same, as they both rely on a very similar algorithm, stock reduction (Kimura et. al. 1984)

\end{itemize}