Fisheries are important both economically and socially, however, they are also a source of conflict since stocks often straddle Exclusive Economic Zones (EEZs) or are conducted in Areas Beyond National Jurisdiction (ABNJ). They may also impact endangered, threatened and protected species (ETP) or vulnerable marine ecosystems (VMEs). There are therefore potentially many stakeholders with  conflicting objectives and divergent views, which may mean that uncertainties are used to support stakeholder positions in order to strengthen or weaken management measures \citep{fromentin2014spectre}.

Fish stocks can not be observed directly and estimates of stock status rely on models with a variety of assumptions, that use fishery-dependent and independent data sets. The quality of data from many small-scale fisheries is insufficient to allow for the application of conventional assessment methods \citep[e.g.][]{dowling2008developing}. In such case data-poor methods, such as those based on catch-only methods are used to estimate productivity and reconstruct historical abundance trends by making assumptions about final biomass relative to unfished and initial biomass \citep[e.g.][]{thorson2012using,froese2017estimating,}. However, simulation testing has indicated that catch-only methods only perform well only when assumptions regarding final relative abundance are met \citep{wetzel2015evaluating}. Despite these reservation estimates of global stock status require assessments of both data poor and data rich stocks are required for strategic planning. There is a need, therefore, for the validation of catch-only methods to increase trust amongst the public, stake and asset-holders and policymakers. %\citep[][]{saltelli2020five}. 

Validation requires model estimates to be compared to known values (i.e. observations) or well estimated historical values \citep{kell2021hindcast}. The only observations used in catch-only-method are catches, and if catch observations are removed then the method can not be run. Therefore to evaluate catch-only methods a reference set of data-rich stocks for a variety of regions, species and fisheries obtained from the RAM legacy database\footnote{RAM Legacy Stock Assessment Database. 2018. Version 4.44-assessment-only. Released 2018-12-22. Accessed [Date accessed 2020-10-30]. Retrieved from DOI:10.5281/zenodo.2542919.} was used. 

We then evaluate the knowledge requirements, in the form of priors for population growth rate ($r$) and initial and final depletion, and the form of the production function for catch-only methods to provide assessments of stock status relative to maximum sustainable yield ($MSY$) targets reference points. We also compare catch-only methods to assessments that use an index of relative abundance for calibration. To do this we use Receiver Operating Characteristic (ROC) curves to evaluate the ability of models and reference points to identify and ranking stocks with respect to being overfished. 

\iffalse
 Many fisheries support the livelihoods of people with few other sources of protein (Hall, Hilborn, Andrew, \& Allison, 2013), and income from fisheries can be a major contributor to social well-being in coastal and inland communities (FAO, 2011). Thirty-one per cent of marine fish stocks globally are estimated to be over-exploited, however, meaning that productivity is lower than what could be supported if fishing pressure was reduced (FAO, 2011).
 
In some parts of the world, management regulations have successfully reduced the capacity of fishing fleets and reduced fishing pressure to levels that should enable stock recovery to levels that could support maximum sustainable yield (Bell, Watson, \& Ye, 2017; Rosenberg et al., 2017; Worm et al., 2009).

In the case of small scale and artisanal fisheries a growing body of research suggests that despite the lack of traditional top-down management by central governments, many of these fisheries have managed to avoid the “tragedy of the commons” where common-pool resources are inevitably degraded (Feeny et al. 1996; Ostrom et al. 1999). At the same time, recent work on community co-management, a widespread approach to the management of small-scale fisheries, has elucidated the characteristics of such systems that lead them to be effective (Gutierrez et al. 2011).

There is, however, currently a lively scientific debate about the global status of marine fisheries. Recent evaluations suggest that globally populations of exploited marine fish and invertebrate have declined 38% between 1970 and 2007, but have on average been stable since the early 1990s (Hutchings et al. 2010).  The information base, however, to make these judgements are limited at best and the methodologies used are mainly qualitative and use expert judgement when evaluating global performance.

The purpose of this proposal is therefore to increase our understanding of the status and trends of fisheries in developing countries with the intention of helping to improve fishery management in S.E. Asia in particular. To do this we propose to evaluate the robustness of data-poor methods used to evaluate global trends in-stock status. To do this we will evaluate the impact of commonly made assumptions when attempting to measure trends in overfishing globally or regionally. We will also evaluate the value-of-information (VoI), i.e. based on the quality of time series of catch and effort, life history parameters and expert knowledge used in the assessments. To do this we will evaluate the robustness of the different data-poor methods used to determine stock status using the RAM legacy database (https://www.ramlegacy.org) and FishBase (https://www.fishbase.de).

The RAM legacy database is a compilation of stock assessment results for commercially exploited marine populations from around the world. The stocks assessment results in the database have been used extensively in management and many global studies (e.g. Hilborn at al., 2020, Rosseau, et al., 2019) have made extensive use of the database. We, therefore, use this well-studied dataset as a benchmark to evaluate the performance of techniques currently used to assess data-poor fish stocks globally that vary in their data and knowledge requirements.

information on all species currently known in the world: taxonomy, biology, trophic ecology, life history, and uses, as well as historical data reaching back to 250 years.

The use of historical data for simulation testing provides an objective way of evaluating assessment methods and provides a scoping exercise by helping to identify scenarios to be used in future simulations and Management Strategy Evaluation. information on all species currently known in the world: taxonomy, biology, trophic ecology, life history, and uses, as well as historical data reaching back to 250 years.

The use of historical data for simulation testing provides an objective way of evaluating assessment methods and provides a scoping exercise by helping to identify scenarios to be used in future simulations and Management Strategy Evaluation.
\fi
