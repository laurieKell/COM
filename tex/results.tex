Estimates of $B:B_{MSY}$ are compared to the RAM legacy values in figure \ref{fig:cf}.  If the COMs were able to assess the stock trends without error the points would lie along the $y=x$ line, while if they were able to classify stocks with respect to $B:B_{MSY}$ then points would either fall in the top right or bottom left quadrants defined by the red lines corresponding to $B:B_{MSY}$. The positive condition (P) is defined as the number of cases where $B \geq B_{MSY}$ in the reference set and the negative condition (N) where $B<B_{MSY}$. The number of real negative cases in the data true positive (TP) correctly classified are termed true negative (TN), while those incorrectly classified are known as false positive (FP) equivalent to a false alarm or Type I error; false negative (FN) are equivalent to a Type II error. Sensitivity ($\frac{TP}{TP+FN}$) measures the proportion of positives that are correctly identified, while specificity ($\frac{TN}{TN+FP}$) measures the proportion of negatives that are correctly identified.

Figure \ref{fig:roc} shows ROC curves for COMs, comparing perfect knowledge for initial and final depletion (i.e. actual) to the heuristics. The blue dark line corresponds to using the heuristics alone without data. Scenarios were run for the shape of the production function (Schaeffer or Fox) and the $r$ prior (low or high). These are compared within a panel for using the known (\laurie{actual}) value for initial and final depletion or the heuristic. The curves were similar within a panel while the points identifying the reference level, based on the estimates $B:B_{MSY}$ and $F:F_{MSY}$vary across scenarios. This shows that the choice of production function and initial prior for $r$ do not have a major effect on ranking but are important for classification.
Only when final depletion was known (first column) did the COM perform well. When the heuristic for final depletion was used the data had no effect.

To evaluate how well a biomass dynamic model with an index as well as catch would perform figure \ref{fig:roc-index} shows the ROC curve for JABBA fitted to a perfect index with a CV of 30\%.

The trends in $B:B_{MSY}$ by stock are shown in figure \ref{fig:ts}. The black lines are the RAM values and the blue and red lines are the estimates using the perfect index, for two production functions \laurie{(add to legend)} and dashed/solid is for the $r$ prior; the horizontal lines indicate 120\%, 100\% and \80 of $B:B_{MSY}$. A range of trends are seen, although all stock has declined some continue to decline, some have stabilised \laurie{(i, ii, ...)} and others have started to recover \laurie{(I, II, ...)}. Another characteristic is inter-annual variability since some stocks \laurie{(A, B, ...)} show smooth trends while others show fluctuations that appear to be independent of fishing \laurie{(X, Y, ...)}








